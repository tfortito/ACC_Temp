%% bare_conf.tex
%% V1.3
%% 2007/01/11
%% by Michael Shell
%% See:
%% http://www.michaelshell.org/
%% for current contact information.
%%
%% This is a skeleton file demonstrating the use of IEEEtran.cls
%% (requires IEEEtran.cls version 1.7 or later) with an IEEE conference paper.
%%
%% Support sites:
%% http://www.michaelshell.org/tex/ieeetran/
%% http://www.ctan.org/tex-archive/macros/latex/contrib/IEEEtran/
%% and

%%********************conca*****************************************************
%% Legal Notice:
%% This code is offered as-is without any warranty either expressed or
%% implied; without even the implied warranty of MERCHANTABILITY or
%% FITNESS FOR A PARTICULAR PURPOSE! 
%% User assumes all risk.
%% In no event shall IEEE or any contributor to this code be liable for
%% any damages or losses, including, but not limited to, incidental,
%% consequential, or any other damages, resulting from the use or misuse
%% of any information contained here.
%%
%% All comments are the opinions of their respective authors and are not
%% necessarily endorsed by the IEEE.
%%
%% This work is distributed under the LaTeX Project Public License (LPPL)
%% ( http://www.latex-project.org/ ) version 1.3, and may be freely used,
%% distributed and modified. A copy of the LPPL, version 1.3, is included
%% in the base LaTeX documentation of all distributions of LaTeX released
%% 2003/12/01 or later.
%% Retain all contribution notices and credits.
%% ** Modified files should be clearly indicated as such, including  **
%% ** renaming them and changing author support contact information. **
%%
%% File list of work: IEEEtran.cls, IEEEtran_HOWTO.pdf, bare_adv.tex,
%%                    bare_conf.tex, bare_jrnl.tex, bare_jrnl_compsoc.tex
%%*************************************************************************

% *** Authors should verify (and, if needed, correct) their LaTeX system  ***
% *** with the testflow diagnostic prior to trusting their LaTeX platform ***
% *** with production work. IEEE's font choices can trigger bugs that do  ***
% *** not appear when using other class files.                            ***
% The testflow support page is at:
% http://www.michaelshell.org/tex/testflow/



% Note that the a4paper option is mainly intended so that authors in
% countries using A4 can easily print to A4 and see how their papers will
% look in print - the typesetting of the document will not typically be
% affected with changes in paper size (but the bottom and side margins will).
% Use the testflow package mentioned above to verify correct handling of
% both paper sizes by the user's LaTeX system.
%
% Also note that the "draftcls" or "draftclsnofoot", not "draft", option
% should be used if it is desired that the figures are to be displayed in
% draft mode.
%
%\documentclass[conference]{IEEEtran}
\documentclass[conference,a4paper]{IEEEtran}
% Add the compsoc option for Computer Society conferences.
%
% If IEEEtran.cls has not been installed into the LaTeX system files,
% manually specify the path to it like:
% \documentclass[conference]{../sty/IEEEtran}





% Some very useful LaTeX packages include:
% (uncomment the ones you want to load)


% *** MISC UTILITY PACKAGES ***
%
\usepackage{graphicx}
\usepackage{ifpdf}
% Heiko Oberdiek's ifpdf.sty is very useful if you need conditional
% compilation based on whether the output is pdf or dvi.
 %usage:
 \ifpdf
   % pdf code
 \else
%   % dvi code
 \fi
% The latest version of ifpdf.sty can be obtained from:
% http://www.ctan.org/tex-archive/macros/latex/contrib/oberdiek/
% Also, note that IEEEtran.cls V1.7 and later provides a builtin
% \ifCLASSINFOpdf conditional that works the same way.
% When switching from latex to pdflatex and vice-versa, the compiler may
% have to be run twice to clear warning/error messages.






% *** CITATION PACKAGES ***
%
\usepackage{cite}
% cite.sty was written by Donald Arseneau
% V1.6 and later of IEEEtran pre-defines the format of the cite.sty package
% \cite{} output to follow that of IEEE. Loading the cite package will
% result in citation numbers being automatically sorted and properly
% "compressed/ranged". e.g., [1], [9], [2], [7], [5], [6] without using
% cite.sty will become [1], [2], [5]--[7], [9] using cite.sty. cite.sty's
% \cite will automatically add leading space, if needed. Use cite.sty's
% noadjust option (cite.sty V3.8 and later) if you want to turn this off.
% cite.sty is already installed on most LaTeX systems. Be sure and use
% version 4.0 (2003-05-27) and later if using hyperref.sty. cite.sty does
% not currently provide for hyperlinked citations.
% The latest version can be obtained at:
% http://www.ctan.org/tex-archive/macros/latex/contrib/cite/
% The documentation is contained in the cite.sty file itself.






% *** GRAPHICS RELATED PACKAGES ***
%
\ifCLASSINFOpdf
  % \usepackage[pdftex]{graphicx}
  % declare the path(s) where your graphic files are
  % \graphicspath{{../pdf/}{../jpeg/}}
  % and their extensions so you won't have to specify these with
  % every instance of \includegraphics
  % \DeclareGraphicsExtensions{.pdf,.jpeg,.png}
\else
  % or other class option (dvipsone, dvipdf, if not using dvips). graphicx
  % will default to the driver specified in the system graphics.cfg if no
  % driver is specified.
  % \usepackage[dvips]{graphicx}
  % declare the path(s) where your graphic files are
  % \graphicspath{{../eps/}}
  % and their extensions so you won't have to specify these with
  % every instance of \includegraphics
  % \DeclareGraphicsExtensions{.eps}
\fi
% graphicx was written by David Carlisle and Sebastian Rahtz. It is
% required if you want graphics, photos, etc. graphicx.sty is already
% installed on most LaTeX systems. The latest version and documentation can
% be obtained at: 
% http://www.ctan.org/tex-archive/macros/latex/required/graphics/
% Another good source of documentation is "Using Imported Graphics in
% LaTeX2e" by Keith Reckdahl which can be found as epslatex.ps or
% epslatex.pdf at: http://www.ctan.org/tex-archive/info/
%
% latex, and pdflatex in dvi mode, support graphics in encapsulated
% postscript (.eps) format. pdflatex in pdf mode supports graphics
% in .pdf, .jpeg, .png and .mps (metapost) formats. Users should ensure
% that all non-photo figures use a vector format (.eps, .pdf, .mps) and
% not a bitmapped formats (.jpeg, .png). IEEE frowns on bitmapped formats
% which can result in "jaggedy"/blurry rendering of lines and letters as
% well as large increases in file sizes.
%
% You can find documentation about the pdfTeX application at:
% http://www.tug.org/applications/pdftex





% *** MATH PACKAGES ***
%
\usepackage[cmex10]{amsmath}
% A popular package from the American Mathematical Society that provides
% many useful and powerful commands for dealing with mathematics. If using
% it, be sure to load this package with the cmex10 option to ensure that
% only type 1 fonts will utilized at all point sizes. Without this option,
% it is possible that some math symbols, particularly those within
% footnotes, will be rendered in bitmap form which will result in a
% document that can not be IEEE Xplore compliant!
%
% Also, note that the amsmath package sets \interdisplaylinepenalty to 10000
% thus preventing page breaks from occurring within multiline equations. Use:
%\interdisplaylinepenalty=2500
% after loading amsmath to restore such page breaks as IEEEtran.cls normally
% does. amsmath.sty is already installed on most LaTeX systems. The latest
% version and documentation can be obtained at:
% http://www.ctan.org/tex-archive/macros/latex/required/amslatex/math/





% *** SPECIALIZED LIST PACKAGES ***
%
\usepackage{algorithmic}
% algorithmic.sty was written by Peter Williams and Rogerio Brito.
% This package provides an algorithmic environment fo describing algorithms.
% You can use the algorithmic environment in-text or within a figure
% environment to provide for a floating algorithm. Do NOT use the algorithm
% floating environment provided by algorithm.sty (by the same authors) or
% algorithm2e.sty (by Christophe Fiorio) as IEEE does not use dedicated
% algorithm float types and packages that provide these will not provide
% correct IEEE style captions. The latest version and documentation of
% algorithmic.sty can be obtained at:
% http://www.ctan.org/tex-archive/macros/latex/contrib/algorithms/
% There is also a support site at:
% http://algorithms.berlios.de/index.html
% Also of interest may be the (relatively newer and more customizable)
% algorithmicx.sty package by Szasz Janos:
% http://www.ctan.org/tex-archive/macros/latex/contrib/algorithmicx/




% *** ALIGNMENT PACKAGES ***
%
\usepackage{array}
% Frank Mittelbach's and David Carlisle's array.sty patches and improves
% the standard LaTeX2e array and tabular environments to provide better
% appearance and additional user controls. As the default LaTeX2e table
% generation code is lacking to the point of almost being broken with
% respect to the quality of the end results, all users are strongly
% advised to use an enhanced (at the very least that provided by array.sty)
% set of table tools. array.sty is already installed on most systems. The
% latest version and documentation can be obtained at:
% http://www.ctan.org/tex-archive/macros/latex/required/tools/


\usepackage{mdwmath}
\usepackage{mdwtab}
% Also highly recommended is Mark Wooding's extremely powerful MDW tools,
% especially mdwmath.sty and mdwtab.sty which are used to format equations
% and tables, respectively. The MDWtools set is already installed on most
% LaTeX systems. The lastest version and documentation is available at:
% http://www.ctan.org/tex-archive/macros/latex/contrib/mdwtools/


% IEEEtran contains the IEEEeqnarray family of commands that can be used to
% generate multiline equations as well as matrices, tables, etc., of high
% quality.


\usepackage{eqparbox}
% Also of notable interest is Scott Pakin's eqparbox package for creating
% (automatically sized) equal width boxes - aka "natural width parboxes".
% Available at:
% http://www.ctan.org/tex-archive/macros/latex/contrib/eqparbox/





% *** SUBFIGURE PACKAGES ***
\usepackage[tight,footnotesize]{subfigure}
% subfigure.sty was written by Steven Douglas Cochran. This package makes it
% easy to put subfigures in your figures. e.g., "Figure 1a and 1b". For IEEE
% work, it is a good idea to load it with the tight package option to reduce
% the amount of white space around the subfigures. subfigure.sty is already
% installed on most LaTeX systems. The latest version and documentation can
% be obtained at:
% http://www.ctan.org/tex-archive/obsolete/macros/latex/contrib/subfigure/
% subfigure.sty has been superceeded by subfig.sty.



\usepackage[caption=false]{caption}
\usepackage[font=footnotesize]{subfig}
% subfig.sty, also written by Steven Douglas Cochran, is the modern
% replacement for subfigure.sty. However, subfig.sty requires and
% automatically loads Axel Sommerfeldt's caption.sty which will override
% IEEEtran.cls handling of captions and this will result in nonIEEE style
% figure/table captions. To prevent this problem, be sure and preload
% caption.sty with its "caption=false" package option. This is will preserve
% IEEEtran.cls handing of captions. Version 1.3 (2005/06/28) and later 
% (recommended due to many improvements over 1.2) of subfig.sty supports
% the caption=false option directly:
\usepackage[caption=false,font=footnotesize]{subfig}
%
% The latest version and documentation can be obtained at:
% http://www.ctan.org/tex-archive/macros/latex/contrib/subfig/
% The latest version and documentation of caption.sty can be obtained at:
% http://www.ctan.org/tex-archive/macros/latex/contrib/caption/



% *** FLOAT PACKAGES ***
%
\usepackage{fixltx2e}
% fixltx2e, the successor to the earlier fix2col.sty, was written by
% Frank Mittelbach and David Carlisle. This package corrects a few problems
% in the LaTeX2e kernel, the most notable of which is that in current
% LaTeX2e releases, the ordering of single and double column floats is not
% guaranteed to be preserved. Thus, an unpatched LaTeX2e can allow a
% single column figure to be placed prior to an earlier double column
% figure. The latest version and documentation can be found at:
% http://www.ctan.org/tex-archive/macros/latex/base/



\usepackage{stfloats}
% stfloats.sty was written by Sigitas Tolusis. This package gives LaTeX2e
% the ability to do double column floats at the bottom of the page as well
% as the top. (e.g., "\begin{figure*}[!b]" is not normally possible in
% LaTeX2e). It also provides a command:
%\fnbelowfloat
% to enable the placement of footnotes below bottom floats (the standard
% LaTeX2e kernel puts them above bottom floats). This is an invasive package
% which rewrites many portions of the LaTeX2e float routines. It may not work
% with other packages that modify the LaTeX2e float routines. The latest
% version and documentation can be obtained at:
% http://www.ctan.org/tex-archive/macros/latex/contrib/sttools/
% Documentation is contained in the stfloats.sty comments as well as in the
% presfull.pdf file. Do not use the stfloats baselinefloat ability as IEEE
% does not allow \baselineskip to stretch. Authors submitting work to the
% IEEE should note that IEEE rarely uses double column equations and
% that authors should try to avoid such use. Do not be tempted to use the
% cuted.sty or midfloat.sty packages (also by Sigitas Tolusis) as IEEE does
% not format its papers in such ways.





% *** PDF, URL AND HYPERLINK PACKAGES ***
%
\usepackage{url}
% url.sty was written by Donald Arseneau. It provides better support for
% handling and breaking URLs. url.sty is already installed on most LaTeX
% systems. The latest version can be obtained at:
% http://www.ctan.org/tex-archive/macros/latex/contrib/misc/
% Read the url.sty source comments for usage information. Basically,
% \url{my_url_here}.





% *** Do not adjust lengths that control margins, column widths, etc. ***
% *** Do not use packages that alter fonts (such as pslatex).         ***
% There should be no need to do such things with IEEEtran.cls V1.6 and later.
% (Unless specifically asked to do so by the journal or conference you plan
% to submit to, of course. )


% correct bad hyphenation here
\hyphenation{op-tical net-works semi-conduc-tor}

\DeclareRobustCommand*{\IEEEauthorrefmark}[1]{\raisebox{0pt}[0pt][0pt]{\textsuperscript{\footnotesize #1}}}

\begin{document}
%
% paper title
% can use linebreaks \\ within to get better formatting as desired
\title{Performance of TURBO Codes over AWGN and BSC Channels}



% author names and affiliations
% use a single column layout for the different affiliations.
% In case the affiliation is too long then use \\ to create multiple lines, as shown in affiliation 4
% Too long lines can result in margin upload warnings ....
\author{\IEEEauthorblockN{
Rohit Gangluy \IEEEauthorrefmark{a}, Artur Baranowski \IEEEauthorrefmark{a}   % 1st author, 1st affiliations
}                                     % ...
%\\
\IEEEauthorblockA{\IEEEauthorrefmark{a}% 1st affiliations
Faculty of Information, Media and Electrical Engineering, Technical University of Cologne, Cologne, Germany}
}

% conference papers do not typically use \thanks and this command
% is locked out in conference mode. If really needed, such as for
% the acknowledgment of grants, issue a \IEEEoverridecommandlockouts
% after \documentclass



% use for special paper notices
%\IEEEspecialpapernotice{(Invited Paper)}




% make the title area
\maketitle


\begin{abstract}
%\boldmath
Turbo Codes is a very powerful error correction technique that made a severe impact on channel coding. In this paper we have computed bit error rate (BER) using Turbo Codes in an Additive Gaussian Noise Channel (AWGN) and Binary Symmetric Channel (BSC) at different block length. 

\textit{Index Terms}—Additive White Gaussian Noise (AWGN), Binary Discrete Memoryless Channel (B-DMC), Binary Erasure Channel (BEC), Binary Symmetric Channel (BSC), Bit Error Rate (BER), Forward Error Correction (FEC), Log Likelihood Ratio (LLR,)MAP, Soft Output Viterbi Algorithm (SOVA).

\end{abstract}
% IEEEtran.cls defaults to using nonbold math in the Abstract.
% This preserves the distinction between vectors and scalars. However,
% if the conference you are submitting to favors bold math in the abstract,
% then you can use LaTeX's standard command \boldmath at the very start
% of the abstract to achieve this. Many IEEE journals/conferences frown on
% math in the abstract anyway.

% no keywords
%{\smallskip \keywords antenna, propagation, measurement.}



% For peer review papers, you can put extra information on the cover
% page as needed:
% \ifCLASSOPTIONpeerreview
% \begin{center} \bfseries EDICS Category: 3-BBND \end{center}
% \fi
%
% For peerreview papers, this IEEEtran command inserts a page break and
% creates the second title. It will be ignored for other modes.
\IEEEpeerreviewmaketitle

\vspace{7pt}
\section{Introduction}
Turbo codes is belong to a class of Shannon-capacity approaching error correcting mechanism. This were introduced by Berrou and Glavieux in ICC93. From all the other class of error correcting codes, Turbo codes can come closer to Shannon’s limit. With relatively low complexity encoding and decoding algorithms, Turbo Codes can achieve their remarkable performance. Here, we check the behavior of block turbo codes or (convolution) turbo codes. The goal of a coding theorists has been to develop codes which are of large equivalent block length and yet contain enough structure that practically decoding is possible. Due to use of pseudo-random inter-leaver, turbo codes appear to be random random to the channel and yet possess structures that decoding is physically realizable. As turbo codes are linear block codes, the encoding operation can be viewed as modulo-2 multiplication of information vector with generator matrix. In Turbo codes, with relatively small block length, we can use the generator matrix with its associated parity check matrix to obtain an upper bound on the probability of code word error. Although the probability of code error provides some insight into the performance of the code, it is slightly irrelevant when one considers the fact that the turbo decoder attempts to minimize the probability of code word error. MAP, log-MAP (SOVA) are used for decoding Turbo codes.

TURBO codes are different from regular convolutional codes because they are systematic. 

% no \IEEEPARstart


\vspace{7pt}
\section{TURBO Codes}

\subsection{The genesis of Turbo Code}
The invention of Turbo Codes finds its origin in the will to compensate for the dis symmetry of the concatenated decoder of fig 1

To do this, a popular feedback technique in Electronics is implemented between the two component Decoder(Fig 2)
\begin{figure}[h!]
    \centering
    \includegraphics{fig}
    \caption{Concatenated Encoder and Decoder}
    \label{fig:my_label}
\end{figure}
\begin{figure}[h!]
\centering
     \includegraphics{fig12.jpg}
    \caption{Decoding the concatenated code with feedback}
    \label{}
\end{figure}
\subsection{Turbo Encoder
}
The 3GPP Turbo encoder (Fig 3) uses a parallel concatenated convolution code. A convolution encoder encodes information sequence (xi) and another encoder encodes the interleaved version of the information sequences. The Turbo encoder has two 8-state constituent encoders and one Turbo code internal interleaver.
\begin{figure}[h!]
\centering
     \includegraphics[scale=0.48]{turboencode.jpg}
    \caption{Turbo Encoder}
    \label{}
\end{figure}
The encoder latency can be calculated using the formula given below
L= D/fMAX
Where D = Encoding Delay
fMAX = System Clock Speed

The termination tail is then appended to the encoded information and utilized as a part of the decoder. The framework is illustrated in Figure 2.We can regard the turbo code as a large piece code. The performance relies on upon the weight circulation - the base distance as well as the quantity of words with low weight. Convolutional codes have usually been encoded in their food forward structure, as (G1,G2)=(1+D2,1+D+D2). 

Notwithstanding, for these codes a solitary 1, i.e. the succession ...0001000..., will give a code word which is exactly the generator vectors and the heaviness of this codeword will in general be low. Plainly a solitary 1 will propagate through any interleaver as a solitary 1, so the conclusion is that on the off chance that we utilize the codes in the food forward structure in the turbo plan the subsequent code will have countless with low weight. The trap is to utilize the codes in their recursive systematic structure where we separate with one of the generator vectors. For example  (1,G2/G1)=1,(1+D+D2)/(1+D2)). 
 



Interleaving: It is a device for reordering a sequence of bits or symbols. A familiar role of interleavers in communications is that of the symbol interleaver which is used after error control coding and signal mapping to ensure that fading bursts affecting blocks of symbols transmitted over the channel are broken up at the receiver by a de-interleaver, prior to decoding
\subsection{Turbo Decoder}
The Turbo decoder (Fig 4) consists of two single soft-in-soft-out (SISO) decoders, which work iteratively. The output of the first decoder will feeds into the next one and thus form a Turbo iteration. Inter leaver and deinterleaver blocks and re-order data in the process.
\begin{figure}[h!]
\centering
     \includegraphics[scale=0.78]{decoder.PNG}
    \caption{Turbo Decoder}
    \label{}
\end{figure}
The Turbo decoder supports the MAXLogMAP decoding algorithm. The algorithm is a simplified version of Log-MAP algorithm. It uses less logical resources and slightly less BER compared to Log-MAP
We can calculate the decoding delay D using one of the following equations:
\begin{itemize}
    \item If K $<$ 264: D = 26 + (2 \times f(K,N) $ $ $+$ $ $14) \times 2 \times I
\end{itemize}
\begin{itemize}
    \item If K $>$ 264, D = 26 + (f(K,N) + 46) \times 2 \times I
\end{itemize}

        
    


where:
\begin{itemize}
    \item K  is the block size
    \item I is the number of decoding iterations
    \item N is the number of engines specified in the decoder
\end{itemize}



f(K,N) = K/N, if K is divisible by N; or f(K,N) = K/8, if K is not divisible by N

In the turbo decoding algorithm, MAP algorithm offers the best performance with iterative decoding, but cannot decode until the decoder receives the entire bit sequence . Due to this BCJR algorithm leads to large decoding delay, power consumption and also required more memory size for iterations . The iterative nature of turbo decoding algorithms leads to increase the complexity compare to conventional FEC decoding algorithms. Two basic iterative decoding algorithms, namely Soft Output Viterbi Algorithm (SOVA) and Maximum A Posteriori Probability (MAP) algorithm demands complex decoding operations over several iteration cycles . Among these algorithms, SOVA has the least computational complexity but the worse BER performance while MAP algorithm provides significantly better BER performance with highest computational complexity. So for implementation of turbo code in real time system, the decoder complexity has to be reduced while preserving BER performance of the system
 




\subsection{Additive White Gaussian Noise (AWGN)
}
AWGN channel’s output can be represented as r(t) = x(t) + n(t) where x(t) belongs to (-1,+1) and n(t) is random noise variable having zero mean.

\begin{figure}[h!]
    \centering
    \includegraphics[scale=0.48]{fig15.jpg}
    \caption{Fig AWGN Channel}
    \label{fig:my_label}
\end{figure}
\subsection{Binary Symmetric Channel(BSC)
}
\begin{figure}[h!]
    \centering
    \includegraphics[scale=0.48]{fig16.jpg}
    \caption{Binary Symmetric Channel(BSC)}
    \label{fig:my_label}
\end{figure}

For BSC, the channel input and output is defined as X ={0,1} and Y = {0,1} respectively
Where the transmitted bit it flipped with a probabliliy of  ‘p’ known as crossover probability by the channel. Transitional probabilities, P(Y = 0 / X = 0) = P(Y = 1 / X = 1) = 1 -p
and P(Y = 1 / X = 0) = P(Y = 0 / X = 1) = p


\subsection{Simulation Parameters}

Both the simulation runs for the AWGN and BSC channels will be executed using the same simulation parameters to ensure comparability. The encoder consists of two identical recursive systematic encoders with code rate $1/2$. Thus the code rate of the TURBO encoder is $1/3$. 

\begin{table}[!htbp]
\centering
\begin{tabular}{|l|l|}
\textbf{Code type }         & Turbo         \\
\textbf{Noise range}        & 1.0 - 2.0 db  \\
\textbf{Noise step}         & 0.1 db        \\
\textbf{Noise type }        & $E_b/N_0$     \\
\textbf{Information bits}   & 1024          \\
\textbf{Codeword size}      & 3072          \\
\textbf{Code rate}          & 1/3           \\
\textbf{Encoder type }      & RSC           \\
\textbf{Interleaver}        & LTE           \\
\textbf{Decoder iterations} & 10            \\
\textbf{Modulation}         & BPSK          \\
\textbf{Puncturer pattern}  & 11, 10, 01 
\end{tabular}
\end{table}

The source will produce 1024 random bits as input.

\vspace{7pt}
\section{AWGN Performance}


\vspace{7pt}
\section{BSC Performance}


\vspace{7pt}
\section{Evaluation}


\vspace{7pt}
\section{Conclusion}



% An example of a floating figure using the graphicx package.
% Note that \label must occur AFTER (or within) \caption.
% For figures, \caption should occur after the \includegraphics.
% Note that IEEEtran v1.7 and later has special internal code that
% is designed to preserve the operation of \label within \caption
% even when the captionsoff option is in effect. However, because
% of issues like this, it may be the safest practice to put all your
% \label just after \caption rather than within \caption{}.
%
% Reminder: the "draftcls" or "draftclsnofoot", not "draft", class
% option should be used if it is desired that the figures are to be
% displayed while in draft mode.
%
%\begin{figure}[!t]
%\centering
%\includegraphics[width=2.5in]{myfigure}
% where an .eps filename suffix will be assumed under latex, 
% and a .pdf suffix will be assumed for pdflatex; or what has been declared
% via \DeclareGraphicsExtensions.
%\caption{Simulation Results}
%\label{fig_sim}
%\end{figure}

% Note that IEEE typically puts floats only at the top, even when this
% results in a large percentage of a column being occupied by floats.


% An example of a double column floating figure using two subfigures.
% (The subfig.sty package must be loaded for this to work.)
% The subfigure \label commands are set within each subfloat command, the
% \label for the overall figure must come after \caption.
% \hfil must be used as a separator to get equal spacing.
% The subfigure.sty package works much the same way, except \subfigure is
% used instead of \subfloat.
%
%\begin{figure*}[!t]
%\centerline{\subfloat[Case I]\includegraphics[width=2.5in]{subfigcase1}%
%\label{fig_first_case}}
%\hfil
%\subfloat[Case II]{\includegraphics[width=2.5in]{subfigcase2}%
%\label{fig_second_case}}}
%\caption{Simulation results}
%\label{fig_sim}
%\end{figure*}
%
% Note that often IEEE papers with subfigures do not employ subfigure
% captions (using the optional argument to \subfloat), but instead will
% reference/describe all of them (a), (b), etc., within the main caption.



% An example of a floating table. Note that, for IEEE style tables, the 
% \caption command should come BEFORE the table. Table text will default to
% \footnotesize as IEEE normally uses this smaller font for tables.
% The \label must come after \caption as always.
%
\begin{table}[!t]
%% increase table row spacing, adjust to taste
%\renewcommand{\arraystretch}{1.3}
% if using array.sty, it might be a good idea to tweak the value of
% \extrarowheight as needed to properly center the text within the cells
%\caption{An Example of a Table}
%\label{table_example}
%\centering
%% Some packages, such as MDW tools, offer better commands for making tables
%% than the plain LaTeX2e tabular which is used here.
\begin{tabular}{|c||c|}
\hline
One & Two\\
\hline
Three & Four\\
\hline
\end{tabular}
\end{table}


% Note that IEEE does not put floats in the very first column - or typically
% anywhere on the first page for that matter. Also, in-text middle ("here")
% positioning is not used. Most IEEE journals/conferences use top floats
% exclusively. Note that, LaTeX2e, unlike IEEE journals/conferences, places
% footnotes above bottom floats. This can be corrected via the \fnbelowfloat
% command of the stfloats package.



% conference papers do not normally have an appendix


% use section* for acknowledgement





% trigger a \newpage just before the given reference
% number - used to balance the columns on the last page
% adjust value as needed - may need to be readjusted if
% the document is modified later
%\IEEEtriggeratref{8}
% The "triggered" command can be changed if desired:
%\IEEEtriggercmd{\enlargethispage{-5in}}

% references section

% can use a bibliography generated by BibTeX as a .bbl file
% BibTeX documentation can be easily obtained at:
% http://www.ctan.org/tex-archive/biblio/bibtex/contrib/doc/
% The IEEEtran BibTeX style support page is at:
% http://www.michaelshell.org/tex/ieeetran/bibtex/
%\bibliographystyle{IEEEtran}
% argument is your BibTeX string definitions and bibliography database(s)
%\bibliography{IEEEabrv,../bib/paper}
%
% <OR> manually copy in the resultant .bbl file
% set second argument of \begin to the number of references
% (used to reserve space for the reference number labels box)
\vspace{7pt}
\begin{thebibliography}{1}


\bibitem{Berrou}
Dr. Claude Berrou, \emph{Turbo Code},Scholarpedia, 5(4):6496.

\bibitem{Vucetic}
Branka Vucetic, \emph{Turbo Codes - Principles and Applications},Springer Science + Media Business,LLC

\bibitem{D.Kene}
Jagdish D.Kene, \emph{Soft-Output Decoding Algorithm For Turbo Codes Implementation},2nd International Conference on Communication,Computing and Security

\bibitem{(Sreenivasulu}
K.Sreenivasulu, K.Aparna,\emph{Turbo Encoding and Decoding Techniques for Secure and Reliable
Data Transmission in Wireless Networks},IJESC,2016

\end{thebibliography}


% that's all folks
\end{document}
